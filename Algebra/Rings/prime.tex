
\documentclass[main.tex]{subfiles}

\begin{document}
 
\section{Prime Ideals}

\begin{definition}
 Let $R$ be a commutative ring. An ideal $P$ is prime if given $xy \in P$ for $x,y \in R$, $x \in P$ or $y \in P$
\end{definition}

\begin{definition}
 An ideal $I$ is called maximal for any ideal $I_m \neq R$ such that $I \subseteq I_m$, then $I_m = R$.
\end{definition}

From these definitions, we can see that maximal ideals are automatically prime ideals

\begin{theorem}
 Every maxmial ideal is prime.
\end{theorem}

\begin{proof}
Let $M$ be a maximal ideal and $x,y$ be elements such that $xy \in M $ but $x \not\in M$. Generate the ideal $M + Ax$ which is an ideal properly contained in $R$. Hence, there exists $c \in R, m \in M$ such that $1 = m + cx$. Mutliplying both sides by $y$ resutlts in the following equality $y = ym + cxy$. Thus, $y \in M$.
\end{proof}

\begin{remark}
 The ideal $\{0\}$ is a prime ideal of $A$ if and only if $A$ is an integral domain
\end{remark}

\begin{theorem}
 If $M$ is a maximal ideal, A/M is a field
\end{theorem}

\begin{proof}
 Let $\bar{x}$ be the residual class for the quotient ring $A/M$. As above, we can write the following equality: $1 = m + rx$ so $rx \in \bar{1}$ class and thus has an inverse. 
\end{proof}

\begin{theorem} (Chinese Remainder Theorem)
Let $R$ be a commutative ring. Let  $a_1,...,a_n$ be ideals such that $(a_i) + (a_j) = (1)$ for $i \neq j$. Given elements $x_1,..,x_n \in R$, there exists $x \in R$ such that $x \equiv x_i \text{ (mod $a_i$) }$ for all $i$
\end{theorem}

\begin{proof}
 For $n=2$, $1 = a_1 + a_2$ and multiplying by $x$ gives us $x = x_1a_1 + x_2a_2$.
 For $i \geq 2$, we can find $a_i \in a_1$ and $b_i \in a_i$ such that $a_i + b_i = 1$. The product $\Pi(a_i+b_i) = 1$ and lies in 
 
 $$ a_1 + \prod_{j \geq 2} a_j = R$$
 
 By the $n = 2$ case, we can find a $y_1$ such that 
 $y_1 = 1 \mod (a_1)$ and $y_1 = 0 \mod \prod a_j$ 
 We can continue this procedure for every $a_i$ to find a corresponding $y_i$ and our $x = y_1x_1 + ... + y_nx_n$.

\end{proof}

We can express the theorem through a succient map as below
$$ R / (\cap_{i=1}^n a_i) \rightarrow R /a_1 \times ... \times R / a_n$$
for pairwise disjoint ideals $a_1...a_n$ in $R$. By the Chinese Remainder Theorem, this map is surjective and hence an isomorphism of rings.

The classical version of the Chinese remainder theorem used ideals generated relatively prime numbers. 

Let $m = \prod_i p_i^{r_i}$ be the factorization of $m$ into primes with exponents $r_i \geq 1$. Then we get the ring isomorphism:

$$\mathbb{Z}/m\mathbb{Z} \cong \prod_i \mathbb{Z}/(p_i^{r_i})\mathbb{Z}$$

Let $R^*$ denote the multiplicative group of units of $R$.

 
\end{document}
