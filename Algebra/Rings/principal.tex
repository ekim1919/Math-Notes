\documentclass[main.tex]{subfiles}

\begin{document}
\section{Principal and Factorial Rings}

From henceforth, we shall assume that the ring $R$ is an integral domain.

\begin{definition}
 An element is $x \in R$ is irreducible if it is not a unit and if it can be written as $x = yz, y,z \in R$, then $y$ or $z$ is a unit. 
\end{definition}

Let $a \neq 0$ and let the principal ideal $(a)$ be prime. It follows that $a$ is irreducible. If we can write $a = bc$, since $(a)$ is prime either $b$ or $c$ lies in the principal ideal. Suppose that $b \in (a)$, then $b = ad$. So we have that 
$a = a(dc)$. Thus, $b$,$c$ must be a unit.

\begin{definition}
 An element $x$ is said to have a unique factorization of irreducible elements if it can be expressed as:
 $$ x = u \prod_{i=1}^n P_i $$
 where $u$ is a unit element.
\end{definition}

By uniqueness, if there exists another factorization $x = u' \prod_{i=1}^n Q_i$, then $P_i = u_iQ_i$ for some unit $u_i$ after permuting the indices.

\begin{definition}
 A ring is called factorial if it is an integral domain and each element exhibits a unique factorization as above. 
\end{definition}

We shall say that an element $a$ divides $b$ if there exists $c \in R$ such that $ac = b$. Likewise, we can define a greatest common divisor by deeming it the element $d \in R$ such that $d|a$ and $d|b$ and any divisor $e$ , $e|d$.

\begin{proposition}
 Let $R$ be a principal entire ring and let $a,b \in A$ with $a \neq b$. Let $(a) +(b) = (c)$, then $c$ is the greatest common divisor of $a$ and $b$. 
\end{proposition}

\begin{proof}
By our assumption, $b = xc$ for some $x \in R$ and similarly $a = yc$ for some $y \in R$. It now suffices to prove that $c$ is the $gcd(a,b)$. Let $d$ be another divisor of $a,b$. then $a = hd$ and $b = jd$ so $r_1a + r_2b = c$ so $r_1hd + r_2jd = c$ or $(r_1h + r_2j)d = c$ as desired.
\end{proof}

\begin{theorem}
 Let $R$ be a principal integral domain. Then $R$ is also factorial.
\end{theorem}

\begin{proof} (Lang)
We must first prove that every element in $R$ as a factorization into irreducible elements. Let $S$ be the set of non-zero principal ideals whose generators do not exhibit such a factorization. Let $(a_1) \in S$ We create a chain of ideals
$$(a_1) \subset (a_2) \subset (a_3).... $$
We argue that such a chain cannot exist as the union of the chain also must be a principal ideal generated by an element $a \in S$. However, $a \in (a_i)$ in some position of the chain. This would cause the chain to stabilize at $(a_i)$. Therefore, any ideal that contains $a$ and is not equal to $a$ has a generator admitting a factorization. We note that $a$ cannot be irreducible by definition.
Thus, it can be written as $a = bc$ and neither can be a unit. Thus, $(a) \neq (c)$ and $(a) \neq (b)$. It follows that $b,c$ do admit factorizations into irreducible elements. However, the product of these factorizations admits a factorization for $a$. This causes the class $S$ of chains to collapse and become empty, contradicting our assumption. 
To prove uniqueness, we first prove a lemma
\begin{lemma}
If $p$ is an irreducible element and $p|ab$  for $a,b \in R$, then $p|a$ or $p|b$
\end{lemma}
$ab = hp \implies h^{-1}ab = p$. This implies that one of the two two terms are units. Multiply both sides by the unit for $a$ then $p|a$. By symmetry, $p|b$.

Suppose that $x = p_1 .... p_n = q_1 ... q_n$ are two factorizations for $x$. Now, by our lemma. $p_1$ must divide one of the irreducibles $q_1,....,q_n$. Suppose, by an appropriate permutation of the indices that $p_1 | q_1$. Since both are irreducibles, $q_1  u_1p_1$ for a unit $u_1 \in R$. Proceed with induction from
$p_2....p_n = q_2...q_n$.
\end{proof}

Expanding on the concept above of uniqueness of factorizations, we could create equivalence classes by the following relation: $a \sim b$ if $a = ub$ for some unit $u \in R$. Create a set $P$ of irreducibles from each equivalence class. Let $\nu_a : R \rightarrow (P \rightarrow \mathbb{N})$ which dictates the powers of the irreducibles in its factorization. Thus, $a$ can be expressed as 

$$ a = u \prod_i P_i^{\nu_a(P_i)} $$

This factorial framework gives us several analogies to number-theoretic concepts.
principal ideals generated by irreducibles $(p)$ are prime ideals. A notion of a least common multiple can be defined as such: let $a_1,..a_n $ be in our factorial ring. The least common multiple would be $$c = \prod_i P_i^{\max_i \{\nu_{a_i}(p_i))} $$ 



\end{document}
