
\documentclass[main.tex]{subfiles}

\begin{document}
\section{Cyclic groups}

\begin{theorem}
 	
	\begin{enumerate}
	\item An infinite cyclic group has exactly two generators (if $a$ is a generator, then $a^{-1}$ is the only other generator).
	\item Let $G$ be a finite cyclic group of order $n$, and let $x$ be a generator. The set of generators of $G$ consists of those powers $x^v$ of $x$ such that $v$ is relatively prime to $n$.
	\item Let $G$ be a cyclic group. and let $a,b$ be two generators. Then there exists an automorphism of $G$ mapping $a$ onto $b$. Conversely, any automorphism of $G$ maps a on some generator of $G$.
	\item Let $G$ be a cyclic group with order $n$ and let $d$ divide $n$. Then there exists a unique subgroup $H$ with order $d$. (Converse of Lagrange Theorem)
	\item Let $G_1$ and $G_2$ be cyclic groups of order $m$,$n$ respectively such that they are relatively prime. Then $G_1 \times G_2$ is cyclic. 
	\item Let $G$ be a finite abelian group. If $G$ is not cyclic, then there exists a prime $p$ and a subgroup of $G$ isomorphic to $C \times C$, where $C$ is cyclic of order $p$.
	\end{enumerate}

\end{theorem}

\begin{proof}
\begin{enumerate}
\item Any infinite cyclic group $G$ is isomorphic to $\mathbb{Z}$ by the mapping $\phi: \mathbb{Z} \rightarrow G$ that takes $\phi(1) = a$ and $\phi(-1) = a^{-1}$. Let $z \in \mathbb{Z}$ such that $z \neq 1,-1$. We know that the elements of the  subgroup generated by $z$ are exactly the integers of the form $nz$ where $n \in \mathbb{Z}$. Hence, $<z> \subset \mathbb{Z}$ is a strict inclusion and does not generate $\mathbb{Z}$. Hence, the image $\phi(<z>)$ is a subgroup which cannot contain all elements of $G$ as $\phi$ is an isomorphism. The assertion immediately follows.

\item  Since $v$ does not divide $n$,  $cv$ cannot be equal any multiple of n for $ 0 \leq c < n$. By Lagrange's Theorem, the cyclic subgroup generated by $x^v$ cannot be a proper subgroup as that would contradict our assumption. Hence, the order of the subgroup generated by $x^v$ must be $n$ i.e $G$.

\item Let $\phi: G \rightarrow G$ be the map such that $\phi(a) = b$. As $\phi(a^n) = b^n$, it follows that $\phi$ is an automorphism (the same can be said for the inverse). To prove the converse, let $\psi \in Aut(G)$. We prove the case for cyclic groups of finite order $n$. Let $\theta: \mathbb{Z}/n\mathbb{Z} \rightarrow G$ be the map $\theta([r]) = a^r$ where $a$ is a generator for $G$. By part 2 above, we know that $Aut(\mathbb{Z}/n\mathbb{Z})$ consists of mappings from generators to generators. Using $\theta$, one can define the isomorphism $\sigma:  Aut(G) \rightarrow Aut(\mathbb{Z}/n\mathbb{Z})$. Thus, 
$\psi = \theta \circ \sigma \circ \theta^{-1}$ which shows that ever automorphism must map $a$ to another generator. 
\end{enumerate}
\end{proof}

\end{document}