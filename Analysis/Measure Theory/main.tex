\documentclass[12pt]{article}
\setlength{\oddsidemargin}{0in}
\setlength{\evensidemargin}{0in}
\setlength{\textwidth}{6.5in}
\setlength{\parindent}{0in}
\setlength{\parskip}{\baselineskip}

\usepackage{amsmath,amsfonts,amssymb}
\usepackage{amsthm}
\usepackage{subfiles}



\newtheorem{theorem}{Theorem}[section]
\newtheorem{question}{Question}[section]
\newtheorem{lemma}[theorem]{Lemma}
\newtheorem{definition}{Definition}[section]
\newtheorem{proposition}{Proposition}[section]

\theoremstyle{remark}
\newtheorem{remark}{Remark}

\title{Notes on Measure Theory}
\author{Edward Kim}



\begin{document}

\maketitle
\clearpage

\section{Preliminaries}
\begin{definition}
A class of subsets $\mathcal{A}$ of subsets of a set $X$ is deemed an algebra if
	\begin{enumerate}
	\item $X \in \mathcal{A}$
	\item $A,B \in \mathcal{A} \implies A \backslash B \in \mathcal{A}$  
	\end{enumerate}
\end{definition}

\begin{remark}
By the following definition, we note that $X\backslash X = \emptyset \in \mathcal{A}$, For any $B \in \mathcal{A}, X \backslash B = \bar{B} \in \mathcal{A}$. Hence, it easily follows that finite unions and intersections of sets in are $\mathcal{A}$.  
\end{remark}

\begin{definition}
An algebra is called a $\sigma$-algebra if for any countable set of subsets $\{\mathcal{A}_n\} \subset \mathcal{A}$, $\bigcup_{\infty} \{\mathcal{A}_n\} \in \mathcal{A}$
\end{definition}

We shall prove that any family $\mathcal{F}$ of subsets of an arbitrary set $X$ generates a unique $\sigma$-algebra.

\begin{theorem}
For any family of subsets $\mathcal{F}$ on a set $X$, there exists a unique minimal m$\sigma$-algebra on $X$.
\end{theorem}

\begin{proof}

Let $$ I = \bigcap_{ \mathcal{F} \subset X} \mathcal{A}_{\mathcal{F}} $$
over all $\sigma$-algebras over $\mathcal{F}$
It immediaately follows that $X \in I$ and $\mathcal{F} \in I$. If $\{A_n\}_{\infty} \subset I$, then $\cup_{n=1}^{\infty} \in I$ as the union is taken over $\sigma$-algebras. IF $A,B \in I$, then $A/B \in I$ as every member contains its difference. Thus, $I$ is a $\sigma$-algebra over $\mathcal{F}$
To show uniqueness, suppose $\mathcal{B}$ is another minimal $\sigma$-algebra over $\mathcal{F}$, then $I \subset \mathcal{B}$. However, this equates to $I = \mathcal{B}$ as $I \cap \mathcal{B}$ is a $\sigma$-algebra and $I \cap \mathcal{B} \subseteq \mathcal{B}$. 
  
\end{proof}

We deem the pair $(X,\mathcal{A})$ of a set $X$ and a $\sigma$-algebra $\mathcal{A}$ on it as a measure space.


\begin{definition}
	Let $\mathcal{A}$ be a $\sigma$-algebra on a set $X$ and let $\mu: \mathcal{A} \rightarrow [0,\infty]$ be a 	function that satisfies that following properties:
	\begin{enumerate}
	\item $\mu(\emptyset) = 0$
	\item $\mu(\cup_{n = 1}^{\infty} \mathcal{A}_n) = \sum_{n = 1}^{\infty} \mu(\mathcal{A}_n)$
\end{enumerate}
\end{definition} 

\begin{definition}
Let  $\mu:2^X \rightarrow [0,\infty]$ be a function such that 
\begin{enumerate}
	\item $\mu(\emptyset) = 0$
	\item $\mu(A) \subseteq \mu(B) \text{ if } A \subset B$ (Monoticity)
	\item $\mu(\cup_{n=1}^{\infty} A_n \leq \sum_{n=1}^{\infty} \mu(A_n) \text{ for } \{A_n\}_{\infty}$(Countable Subadditivity)
\end{enumerate}	

Such a function is called an outer measure on $X$
\end{definition}

\begin{definition}
Let $\mu$ be an outer measure on $X$. A set $A$ is $\mu$-measureble if for all $E \subset X$
$$ \mu(E) = \mu(E \cap A) + \mu(E\backslash A) $$
\end{definition}

\begin{theorem}(Catheodory's Theorem)
\label{cath}
The set of $\mu$-measurable sets on $X$ form a $\sigma$-algebra. Hence $\mu$ defines a measure on the set of $\mu$-measurable sets as described.
\end{theorem}

\begin{proof}
Denote the $\mu$-measurable sets as $\mathcal{M}$. We immediately observe that $X \in \mathcal{M}$. Let $A,B \in \mathcal{M}$ and let $E$ be a subset of $X$, then the following holds:
$$ \mu(E) =  \mu(E \cap A) + \mu(E\backslash A) =  \mu(E \cap A) + \mu((E \backslash A) \cap B) + \mu((E \backslash A) \backslash B) =$$
$$ \mu((E\backslash (B \backslash A) \cap A) + \mu((E \backslash (B \backslash A)) \backslash A) + \mu(E \cap B \backslash A) = \mu(E \backslash (B \backslash A)) + \mu(E \cap B \backslash A)$$


Thus, $A \backslash B \in \mathcal{M}$

To show that $\mathcal{M}$ is closed under countable union, we first note that for $A = \cup_{n =1}^{\infty} A_n, A_n \in \mathcal{M}$

$$\mu(E) \leq \mu(E \cap A) + \mu(E \backslash A)$$
by countable subadditivity. To show the converse,

let $B_n = A_n \backslash (\cup_{i=1}^{i-1} A_i)$. As above $B_i \in \mathcal{M}$ and

$$ \mu(E \cap \bigcup_{i \in \mathbb{N}} B_i) \leq \sum_{i \in \mathbb{N}} \mu(E \cap B_i) \leq \sum_{i \in \mathbb{N}} \mu(E \cap A_i)   $$

follows by countable subadditivity and monoticity.

Thus, for any $A_n$,

$$\mu(E) = \mu(E \cap A_n) + \mu(E \backslash A_n) \geq \mu(E \cap A) + \mu(E \backslash A) $$

The inequality $\mu(E\backslash A_n) \geq \mu(E \backslash A)$ follows from monoticity.
Hence, $A \in \mathcal{A}$ as desired.

\end{proof}

\begin{theorem} Let $\mathcal{A}$ be a $\sigma$-algebra on a set X. Define $\mu^*: 2^X \rightarrow [0,\infty]$ as
$$ \mu^*(E) = \inf\{\sum_{n \in I} \mu(A_i)  : \{{A_i}\}_{I} \subset \mathcal{A}, B \subset \bigcup_{i \in I} A_i, I \subset \mathbb{N} \}$$ 

then $\mu^*$ is an outer measure.

\end{theorem}

\begin{proof}
From our definition, it is clear that $\mu(\emptyset) = 0$
\newline
Now let $E_1 \subset E_2 \subset X$.
then $\cup_i\{A_{1,i}\}_{I_1,E_1} \subseteq \cup_i\{A_{2,i}\}_{I_2,E_2}$ when taken over index sets $I_1$ and $I_2$
such that $$\sum_{n \in I_1} \mu(A_{1,n})  < \mu^*(E_1) + \epsilon$$
$$\sum_{n \in I_2} \mu(A_{2,n})  < \mu^*(E_2) +  \epsilon$$

(We wish to show the inequality for monoticity in general. We must force the index sets to sufficinently cover the subsets without extraneous members to render the inequality false for some cases. Find a way to prove this rigorously)

Finally, countable subadditivity can be shown by noticing that for $\mu$:

$$ \mu(\bigcup_{i = 1}^{\infty} A_i) \leq \sum_{i=1}^{\infty} \mu(A_i), A_i \in \mathcal{A} $$
Taking the infimum on both sides yields the desired result:
$$  \mu^*(\bigcup_{i = 1}^{\infty} A_i) \leq \sum_{i=1}^{\infty} \mu^*(A_i), A_i \in \mathcal{A} $$
\end{proof}

Thus, $\mu^*$ is indeed an outer measure.

\begin{definition}
An outer measure $\mu^*$ is called regular if for any $A \subset X$, there exists a $\mu^*$-measureable set $B \subset X$ such that $\mu^*(A) = \mu^*(B)$. 
\end{definition}

\section{Borel and Baire spaces}

\subsection{Borel Spaces and Other Preliminaries}

\begin{definition}(Borel space)
Let $X$ be a set endowed with a topology $\mathcal{T}$. The Borel space $\mathcal{B}$(X) is the $\sigma$-algebra generated by the open sets of $X$.
\end{definition}

\begin{definition}
Let $X,Y$ be topological spaces. A map $f: X \rightarrow Y$ is Borel measurable if for $B \in \mathcal{B}(Y)$, $f^{-1}(B) \in \mathcal{B}(X)$. Extending this to any two measure spaces $(X,\theta),(Y,\delta)$, $f$ is measurable (with respect to $\theta$ and $\delta$) if $f^{-1}(X) \in \theta, \text{ for all } X \in \delta $
\end{definition}

\begin{lemma}
Let $f: X \rightarrow Y$ be a continuous map between topological spaces $X,Y$. Then $f$ is Borel measurable,
\end{lemma}

\begin{proof}
 For any open set $\mathcal{O} \subset Y$, $f^{-1}(\mathcal{O}) \subset X$ is also an open set in $X$. By definition, any $\mathcal{O} \in \mathcal{B}(Y)$ is an open set, thus  $f^{-1}(\mathcal{O}) \in \mathcal{B}(X)$. We can prove a stronger result by showing that the subsets $B \in \mathcal{B}(Y)$ for $f^{-1}(B) \in \mathcal{B}(X)$ generate $\mathcal{B}(Y)$. Define the set:
 $$\mathcal{S} = \{ B \in \mathcal{B}(Y) : f^{-1}(B) \in \mathcal{B}(X) \} $$
 
Clearly, $Y \in \mathcal{S}$. let $X,Y \in \mathcal{S}$. then it follows that 
$f^{-1}(X) \backslash  f^{-1}(Y) = f^{-1}(X \backslash Y) \in \mathcal{B}(X)$. Hence, $X \backslash Y \in S$.

To show that $\mathcal{S}$ is closed under countable unions, let $\{B_i\}_i \subset \mathcal{S}$ be a countable set. It follows that \
$$ f^{-1}(\bigcup_i B_i) = \bigcup_i f^{-1}(B_i) \in \mathcal{B}(X) $$. Thus, 
$ \bigcup_I B_i \in \mathcal{S}$ and $S$ is a $\sigma$-algebra. Since, $S \subset \mathcal{B}(Y)$, it follows that $S = \mathcal{B}(Y)$.
\end{proof}

We recall the definition of the product topology.

\begin{definition}
Let $(X_i,\mathcal{T}_i)$ be a collection of topological spaces. The product topology is defined as the topology on 
$$ X = \prod_i X_i $$
such that the projections $\pi_i: X \rightarrow X_i$ are continuous i.e the minimal topology generated by sets $\pi_i^{-1}(U_i) : U_i \in \mathcal{T}_i$. 
\end{definition}

We can easily show that this is indeed a product in the category of topological spaces.

\begin{definition}
Let $X$ be a set, $Y$ a topological space and $F = \{f:X \rightarrow Y \}$ be a collection of set-theoretic mappings. The topology of pointwise convergence in $F$ is defined as the product topology on $Y^{|X|}$ such that $x \in X$ is identified by the element $(y_x)$ such that $f: x \mapsto y_x$ i.e for projections $\pi_i$, we identify $x \in X$ by the unique map $\phi: X \rightarrow \prod_{|X|} Y$ such that $f = \pi_i \circ \phi$ for all $f$.
\end{definition}

\begin{proposition}
 A sequence of functions $f_i$ that converges to $f \in F$ in the pointwise topology if and only if $f_n(x) \rightarrow f(x)$ for all $x \in X$.
\end{proposition}


\subsection{Baire Spaces}

\begin{definition}
 Let $X$ be a topological space. The Baire $\sigma$-algebra is the smallest $\sigma$-algebra generated by the sets
 $$ \{ f^{-1}((0,\infty)) : f \in \mathcal{C}(X)\}$$ where $\mathcal{C}(X)$ is defined as the set of all continuous real-valued functions $f:X \rightarrow \mathbb{R}$.
\end{definition}

Thus, the Baire $\sigma$-algebra is the smallest $\sigma$-algebra such that every continuous real-valued function is measurable.

\section{Polish Spaces and Souslin Sets}

\begin{definition}
 A topological space $X$ is deemed a Polish space if it is homeomorphic to a complete separable metric space.
\end{definition}

\begin{definition}
 A subset of  a Hausdorff space $X$ is Souslin if it is the image of a Polish space under a continuous mapping. $\mathcal{S}(X)$ is the set of all Souslin sets in $X$. A Hausdorff space is Souslin if itself is a Souslin set.
\end{definition}

\begin{definition}
 Define the space $\mathbb{N}^{\infty}$ as the space of all countably infinite sequences of natural numbers. We can define a metric on this space by the following function: for any two sequences $(x_i),(y_i)$, the function $\rho: \mathbb{N}^{\infty} \rightarrow [0,\infty), \rho((x_i),(y_i)) = 2^{-\inf \{j : x_j = y_j\}}$ defines a metric on $\mathbb{N}^{\infty}$.
\end{definition}

\begin{theorem}
 $\mathbb{N}^{\infty}$ is a Polish space.
\end{theorem}

\begin{proof}
 From the metric above, we can construct a countable dense set as follows:
 $$\mathcal{D} = \{ (a_i) \vert a_i = 0, i \geq n, n \in \mathbb{N} \} $$
 To see that $\mathbb{N}^{\infty}$ is complete, let $\{(a_i)\}$ be a Cauchy sequence in $\mathbb{N}^{\infty}$. Thus, for every $2^{-m}$, there exists a $n \in \mathbb{N}$ such that for any $i,j \geq n$, $\rho((a)_i,(a)_j) \leq 2^{-m}$. By our definition, above, this implies that all $(a)_i,(a)_j$ for $i,j \geq n$ $(a)_j, (a)_i$ must share the first $m$ elements of their sequences. Formally, denote position $j$ in a sequence $(a_i) \in \mathbb{N}^{\infty}$ by $(a_i^j)$.
 Then $(a^r)_i = (a^r)_j$ for all $1\leq r \leq m$. Now define $(a)_c$ as follows:
 $$(a)_c^n = (a^n)_i \text{ for $i \geq n$ as above}$$
 
 We see that $\rho((a)_i,(a)_c) \rightarrow 0$ as $i \rightarrow 0$. Thus, $\mathbb{N}^{\infty}$ is complete.
\end{proof}

\begin{proposition}
 $\mathbb{N}^{\infty}$ is homeomorphic to the irrational numbers in $(0,1)$ equipped with the usual topology on $\mathbb{R}$.
\end{proposition}

\begin{proof}
 It will suffice to create a homeomorpism from $\mathbb{Z}^{\infty}$ to $(0,1) \backslash \mathbb{Q}$. We first enumerate the rationals $q_n$ between $(0,1)$ and construct a family of intervals $I_{(a_j)}, (a_j) \in \mathbb{Z}^{\infty}$ as follows:
 \begin{enumerate}
  \item $(0,1) \backslash \mathbb{Q} \subset \bigcup_{s_1,...,s_i \in \mathbb{Z}} I_{(s_1,...,s_i)}$ for all $i \in \mathbb{N}$
  \item The right endpoint of $I_{s_1,...,s_i}$ is the left endpoint of $I_{s_1,...,s_i+1}$ for all $i \in \mathbb{N}$
  \item Let $m$ be the Lebesgue measure on $\mathbb{R}$. Force $m(I_{s_1,...,s_i}) < \frac{1}{i}$ for all $i \in \mathbb{N}$.
  \item $ \bigcup_{s_{i+1} \in \mathbb{Z}} I_{s_1,...,s_{i+1}} \subset I_{s_1,...,s_i}$ for all $i \in \mathbb{N}$.
  \item $q_i$ is the endpoint of some interval $I_{s_1,...,s_i}, i \in \mathbb{N}$
 \end{enumerate}
 
  Now constuct $\phi: \mathbb{Z}^{\infty} \rightarrow (0,1) \backslash \mathbb{Q}$
  as follows:
  $$h((a_n)_n) = \bigcap_{n \in \mathbb{N}} I_{s_1,...,s_n} $$

Note that the intersection is nonempty by the fourth criterion. By the third criterion, the inductively-constructed intersections must be singletons since the length of the intervals go to zero.By the fifth criterion, it the singletons must exclude the rationals. The second criterion ensures the injectivity of $\phi$. Also. for every irrational $r \in (0,1) \backslash \mathbb{Q}$, there exists a $(a_n) \in \mathbb{Z}^{\infty}$ such that $h(a_n) = z$, proving surjectivity. Hence $\phi$ is a bijection that takes $h(\{s_1\} \times ... \times \{s_i\} \times \mathbb{Z}^{\infty}) = I_{s_1,...,s_i} \cap (0,1) \backslash \mathbb{Q}$ and both form basis' in their respective topologies by the first criterion and by the definition of $\phi$.
\end{proof}

\begin{proposition}
Every Polish space is the continous image of $\mathbb{N}^{\infty}$.
\end{proposition}

\begin{proof}
Let $X$ be a Polish space endowed with the metric $\rho$. Construct the following closed balls inductively as follows:
Let $\mathcal{D}$ be the countable dense set in $X$ and let $C_i$ be the closed ball $\bar{B}(x_i,1)$ around $x_i \in \mathcal{D}$ for some enumeration of $\mathcal{D}$. Let $\{x_{i_1,...,i_n,r}\}_{r \in \mathbb{N}}$ be dense in $C_{i_1,...,i_n}$ and define nonempty closed sets $\bar{B}(x_{i_1,...,i_n,r},2^{-n}) \cap C_{i_1,...,i_n}$. Thus, we have the following descending chain of inclusions: $$ ... \subset C_{i_1,...,i_n} \subset ... \subset C_{i_1,i_2}\subset C_{i_1}$$.
Define the map $f:\mathbb{N}^{\infty} \rightarrow X$:
$$f((a)) = \bigcap_{(a) \in \mathbb{N}^{\infty}} C_{(a)} $$
Since $X$ is complete and $(a)$ idenitifies with a unique Cauchy sequence, $f$ is injective and well-defined. Let $(a)_j,(a)_k \in \mathbb{N}^{\infty}$ such that $d((a)_j,(a)_k) \leq 2^{-m}$ and $m > 1$. However, by definition, $f((a)_j),f((a)_k) \in C_{i_1,...,i_{m-1}} \subset \bar{B}(x_{i_1,...,i_{m}},2^{-m+1})$. Thus, $\rho(f((a)_j),f((a)_k)) \leq 2^{-m+1} = 2 \cdot d((a)_j,(a)_k)$. We see that by bounding the metric by $d$, we force the continuity of $f$. By density and our construction above, $f(\mathbb{N}^{\infty}$. To see this, let $x \in X$ and $U_x$ a neighborhood of $x$. By the density, there exists $d \in \mathcal{D}$ such that $d \in U_x$. In particular, there exists a $\bar{B}(d,2^{-m}) \subset U_x$ for a sufficiently small $m$. Hence, there must exist $(a) \in \mathbb{N}^{\infty}$ by following this argument for all neighborhoods $U_x$.
\end{proof}

We will now visit a technique using Souslin sets to manipulate Borel sets from their source Polish space. 

\begin{theorem}
Let $Y$ be a Hasdorff space and let $\{C_i\} \subset \mathcal{S}(Y)$ be disjoint Souslin sets. Then there exists two disjoint Borel sets $B_i \subset \mathcal{B}(Y)$ such that $C_i \subset B_i$.
\end{theorem}





\end{document}
