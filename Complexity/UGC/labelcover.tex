\section{Label-Cover and Projection Games}
%Definition of Label-Cover
We now introduce a problem which manages to provide a natural paradigm for capturing the essence of CSPs and proving inapproximability results. These ``projection games" were introduced by Bellare, Goldreich, and Sudan \cite{bellare1998free}. The $\NP$-hardness of the gap problem version of Label Cover was used by H\aa stad to show tight inapproximability results for $\mathsf{Max3SAT}$ and $\ELin$ \cite{haastad2001some}.

\begin{definition}
A \emph{Label Cover (LC) Problem} instance $\mathcal{G}$ is defined by a bipartite graph $(A \sqcup B,E)$, finite alphabets $\Sigma_A, \Sigma_B$, and a set of projections $\pi_e:\Sigma_A \rightarrow \Sigma_B$ for every edge $e \in E$. Define an \emph{assignment} as consisting of two maps $\mathfrak{A}: A \rightarrow \Sigma_A$, $\mathfrak{B}: B \rightarrow \Sigma_B$. An edge $e = (a,b) \in E$ is said to be satisfied by this assignment if the assignment is compatible with projection $\pi_e$:

\begin{equation}
  \pi_e(\mathfrak{A}(a)) = \mathfrak{B}(b)
\end{equation}

The value of this game will be

\begin{equation} \label{optvalLC}
  \mathsf{Opt}(\mathcal{G}) = \max_{(\mathfrak{A},\mathfrak{B})} \mathbb{E}_{e \sim E}[e \text{ satisfied}]
\end{equation}
In other words, the value will be the largest fraction of edges satisfied by any assignment to the vertices. The corresponding gap problem for Label Cover, $\mathsf{Gap}_{\alpha,\beta}\mathsf{LC}$, is defined as the promise problem:
%
\begin{align*}
    \mathsf{YES} & = \{\mathcal{G} \mid \mathsf{Opt}(\mathcal{G}) \geq \beta\} \\
    \mathsf{NO} & = \{\mathcal{G} \mid \mathsf{Opt}(\mathcal{G}) < \alpha \}
\end{align*}

In the case of perfect completeness, we abbreviate $\mathsf{Gap}_{\alpha,1}\mathsf{LC}$ as simply $\mathsf{Gap}_{\alpha}\mathsf{LC}$.
\end{definition}

There are a few observations worthy of mentioning here. The first regards a type of equivalence between CSP instances and Label Cover instances.
%
Specifically, let $\I$ be an instance of a given $r$-ary CSP $\Psi$ over domain $\Omega$. We can translate this CSP instance into a Label Cover instance as follows:
%
Let the left-hand partition $A$ of our bipartite graph be indexed by the set of constraints $(S,\psi)$ and the right-hand partition $B$ be indexed by the variables of the CSP $V$.
%
Draw an edge from a constraint tuple $(S,\psi)$ to a variable $v$ if that variable appears in $S$.
%
Set $\Sigma_A = \{(q_1,\cdots,q_r) \in \Omega^r \mid \exists \psi \in \Psi, \; \psi(q_1,\cdots,q_r) = 1 \}$
%
and $\Sigma_B = \Omega$, and for every edge $e = ((v_1,\cdots,v_r), \psi), v)$ define the projection $\pi_e:\Sigma_A \rightarrow \Sigma_B$ to be

\[ \pi_e(\omega_1, \cdots, \omega_r) =  \omega_i \text{ if } v_i = v\] \newline

%Finish this.

On the other hand, every Label Cover instance can be seen as a $2$CSP over a sufficiently large domain: the predicates of the CSP would be all $2$-ary predicates $\pi:\Sigma_A \times \Sigma_B \rightarrow \{0,1\}$ representing every possible map from $\Sigma_A \rightarrow \Sigma_B$. Thus, the domain of our CSP can be defined as $\Omega = \Sigma_A \cup \Sigma_B$. The corresponding instance of this CSP would be $(S,\pi_e)$ where $\pi_e$ represents the predicate corresponding to the edge $e$'s projection map $\pi_e$ and $S = (a,b)$ would be the vertices of $e$ between $A$ and $B$ respectively. \newline

By Theorem \ref{equipcpcsp} and the observations made above, there is a method of converting the PCP for $\NP$ given by the PCP Theorem to a CSP, then converting that CSP into a Label Cover instance. This shows that $\mathsf{Gap}_\alpha\mathsf{LC}$ for some $\alpha > 0$ must be $\NP$-hard:

\begin{theorem} (Weak Projection Games Theorem)
 $\mathsf{Gap}_\alpha\mathsf{LC}$ for some $\alpha > 0$ is $\NP$-hard.
\end{theorem}

The primary drawback of this theorem is that it does give us an $\NP$-hardness result for \emph{every} constant $\alpha > 0$. This will be addressed by Raz's Parallel Repetition Theorem.
%examples.
%trivial NP hardness proof.

%\subsection{}
%Arora, Barak

\subsection{Raz's Parallel Repetition Theorem}
In this section, we give a cursory outline of Raz's Parallel Repetition Theorem and its consequences. First, we can extend Definition \ref{pcpdef} to include proof strings over non-boolean alphabets.
%
To accomodate such verifiers, we can extend the definition to $\PCP^\Sigma_{\beta, \alpha}(r(n),q(n))$, the class of languages with polynomial-time verifiers taking proof strings $\pi$ over non-boolean alphabet $\Sigma$ and uses $r(n)$ random bits and at most $q(n)$ queries to $\pi$.
%
The following theorem shows that one can reduce the number of queries to two at the expense of a weaker soundness constant and a larger alphabet:
%
\begin{theorem} \label{2queryinclusion}
  $\PCP^\Sigma_{1 - \alpha, 1}(r(n),q(n)) \subseteq \PCP^{\Sigma^q}_{1 - \frac{\alpha}{q},1}(r(n) + \log{q(n)}, 2)$
\end{theorem}
%
\begin{proof}
Given a verifier for a language $L \in \PCP^\Sigma_{\beta, 1 - \alpha}(r(n),q(n))$, it behaves as such on input $x \in \{0,1\}^n$:
%
\begin{enumerate}
  \item Flips a $r(n)$ coins which we denote as $R$.
  \item Using $R$, it computes a series of indices $i_1, \cdots i_q$ where $q = q(n)$.
  \item It queries $\pi_{i_1}, \cdots \pi_{i_q}\in \Sigma$ from the proof $\pi$.
  \item Finally, it feeds the symbols in a predicate $V_{x,R}(\pi_{i_1}, \cdots \pi_{i_q})$ which outputs ``accept" or ``reject".
\end{enumerate}
We extend this to a modified verfier $V'$ to accomodate the concatenation of two proof strings: if $m = |\Sigma|$ $\pi_1:[m]^q$ then $\rightarrow \Sigma^{q}$ and $\pi_2:[m]\rightarrow \Sigma$:
%
\begin{enumerate}
  \item Flips a $r(n)$ coins which we denote as $R$.
  \item Using $R$, it computes a series of indices $i_1, \cdots i_q$ where $q = q(n)$
  \item Queries $\pi' = \pi_1(i_1, \cdots, i_q)$
  \item Computes the predicate:
  \[ V_{x,R}'(\pi') = V_{x,R}(\pi'_1,\cdots,\pi'_q) \]
  \item Finally, chooses random $\ell \in [q]$ and checks if $\pi_2(i_\ell) = \pi'_\ell$.
  \item The verifier accepts iff both checks pass.
\end{enumerate}

For completeness, if there exists a proof $\pi$ which the original verifier $V$ accepts with probability one, then setting $\pi_1(i_1, \cdots, i_q) = (\pi_{i_1},\cdots,\pi_{i_q})$ and $\pi_2 = \pi$ will induce $V'$ to accept with probability one by construction.
%
As for soundness, suppose  at least $\alpha$ fraction of the predicates $V_{x,R}$ reject in respect to some proof string $\pi$. Each one of the predicates contained in this fraction must fail the check at Step $4$ if $\pi' = \pi_1(i_1, \cdots, i_q) = (\pi_{i_1},\cdots,\pi_{i_q})$. However, $\pi_1$ does not have to respect this rule for any $q$-tuple of queries $(i_1, \cdots, i_q)$ given as input. If so, there must exist at least one index $\ell$ such that $\pi'_\ell \neq \pi_2(i_\ell) = \pi_\ell$. Since such an index is chosen with probability $\frac{1}{q}$, we have that $V'$ must reject with probability at least $\frac{\alpha}{q}$. This completes the proof.
\end{proof}


%
Raz's Parallel Repetition Theorem constructed a verifier which allowed the soundness constant to drop exponentially with the cost of bloating the alphabet size:
\begin{theorem} (Raz's Parallel Repetition Theorem \cite{raz1998parallel}) \label{razparallelrep}
  For all $s \in (0,1)$, there exists $c_s \in (0,s)$ such that:
    \[ \PCP^{\Sigma}_{s,1}(r,2) \subseteq \PCP^{\Sigma^t}_{c_s^t,1}(rt,2)  \]
\end{theorem}

\begin{corollary} \label{razcorollary} For all $\epsilon > 0$, there exists an alphabet $\Sigma$ such that $|\Sigma| \leq \mathsf{poly}(\frac{1}{\epsilon})$:
  $$ \NP \subseteq \PCP^{\Sigma}_{\epsilon,1}(O(\log{n}\cdot\log{1/\epsilon}),2) $$
\end{corollary}
%
\begin{proof}
  For some constant $Q > 0$,
  \begin{align*}
    \NP & \subseteq \PCP_{\frac{1}{2},1}^{\{0,1\}}(O(\log{n}), Q) \\
    & \subseteq \PCP_{1- \frac{1}{2Q},1}^{\{0,1\}^Q}(O(\log{n}), 2) \\
    & \subseteq \PCP_{c_Q^t,1}^{\{0,1\}^{Qt}}(O(t\log{n}), 2) \\
    & \subseteq   \PCP_{\epsilon,1}^{\Sigma}(O(\log{n}\cdot\log{1/\epsilon}), 2)
  \end{align*}
  where $|\Sigma| \leq \mathsf{poly}(\frac{1}{\epsilon})$. The second inclusion follows from Theorem \ref{2queryinclusion} and the third inclusion follows from Theorem \ref{razparallelrep}.
\end{proof}


By combining Corollary \ref{razcorollary}, the $\NP$-hardness of Label Cover comes to fruition:

\begin{theorem} (Projection Games Theorem) \label{labelcoverhard}
  For every $\epsilon > 0$, there exist alphabets $\Sigma_A, \Sigma_B$ where $|\Sigma_A|,|\Sigma_B| \leq \mathsf{poly}(\frac{1}{\epsilon})$ such that $\mathsf{Gap}_\epsilon\mathsf{LC}$ is $\NP$-hard.
\end{theorem}

This is shown by embedding the query pairs of the verifier into the projection constraints of a Label Cover instance graph.
