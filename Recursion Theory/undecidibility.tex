\documentclass[main.tex]{subfiles}

\begin{document}
\section{Undecidibility}

\subsection{Halting Problem}
Given any register program $P$ consisting of a finite alphabet($\mathbb{A}$), we can assign a code to represent the program through a G\"{o}del numbering scheme. Let $\xi_P$ be the G\"{o}del number of $P$.
Since we have a finite alphabet and every program can be represented as a finite string of symbols, the number of possible programs must be countable. Thus, the set 
$\Pi = \{\xi_P \vert \text{$P$ is a valid program over $\mathbb{A}$}\}$ is a countable set. Checking the syntax of programs is computable through a set of rules governing said syntax. Thus, it must be said that
$\Pi$ is a computable set. We now arrive at on of the cornerstone theorems in Computability Theory:

\begin{theorem}
 (Halting Problem) \n
 The set $\Pi_{halt}' = \{\xi_P \vert \text{$P$ is a valid program over $\mathbb{A}$ and $\xi_P \map halt$}\}$ is not computable.
\end{theorem}

\begin{proof}
 Suppose that there existed a program $P_0$ which decided this set.  Then for all $P$
 \begin{gather*}
  P_0: \xi_P \map\square \text{ if } P: \xi_P \map halt \\
  P_0: \xi_P  \map \eta \text{ for some $\eta \neq \square $ if } P: \xi_P \map \infty
 \end{gather*}

 Create any program $P_1$ that does the direct opposite of $P_0$:
 \begin{gather*}
  P_0: \xi_P \map\infty \text{ if } P: \xi_P \map halt \\
P_0: \xi_P \map halt \text{ if } P: \xi_P \map \infty
\end{gather*}

 Now suppose we plugged in $P_1$ into itself, then 
 $P_1:\xi_P \map\infty $ iff $P_1:\xi_P \map halt$. This is a contradiction.
\end{proof}

\begin{lemma}
The set $\Pi_{halt} = \{\xi_P \vert \text{$P$ is a valid program over $\mathbb{A}$ and $\square \map halt$}\}$ is not computable.
\end{lemma}

\begin{lemma}
 $\Pi_{halt}$ is recursively enumerable.
\end{lemma}

\begin{proof}
 Enumerate through $n$=1,2,3.... and generate all programs whose G\"{o}del numbers are $\leq n$. Run each one for at most $n$ steps with input $\square$. If a program in the list halts within that time,
 enumerate the program.
\end{proof}

 We can think of the halting set through a different lens. Let $\phi_e$ denote the $e^{th}$ program. We can also enumerate the domains of $\phi_0,\phi_1,...$ through a similar tactic as the above lemma.
 Let $W_0,W_1,...$ denote the domains of these functions. Since not all of these functions are total, we see that $W_e \subseteq \NN$. The halting set can be described as $K = \{x \vert x \in W_x\}$. This is akin to taking all programs which halt on their own G\"{o}del number.
\subsection{Undecidibility of First-Order Logic}

\begin{theorem}
 (Church's Theorem)
 Let $\phi$ be a first-order sentence. Then the set $\{\phi \vert \mo \phi\}$ of valid first-order sentences is not computable.
\end{theorem}

\subsection{Undecidibility of Arithmetic}


\newpage
\section{Peano Arithmetic and its Models}


\subsection{Encoding Proofs into PA}


\end{document}
